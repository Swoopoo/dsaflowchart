\chapter{Patzer / Bruchtest}
Der Prüfwurf ist \textbf{genauso} erschwert wie der Originalwurf!\\
\spp{\section{Nahkampf}
\begin{center}
    \begin{tabularx}{\linewidth}{cX}
        \hline
        \textbf{2W6}& \textbf{Auswirkungen}\\
        \hline
        2& \textbf{Waffe zerstört: INI -4}. Bei \textbf{BF} 0 -> gilt als 9-10. Bei Faust/Fuß -> gilt als 12.\\
        \hline
        3-5& \textbf{Sturz: INI -2}. Gilt als \textit{am Boden}. Benötigt
        eine Aktione \textit{Position} und eine \textbf{GE}+BE Probe.
        \textit{Standfest}/\textit{Balance}, wandelt in eine einfache
        \textbf{GE}+BE Probe.\\
        \hline
        6-8& \textbf{Stolpern: INI -2}.\\
        \hline
        9-10& \textbf{Waffe verloren: INI -2}. Waffe wiederfinden: Aktion \textit{Position} und \textbf{GE}. Bei Faust/Fuß -> 3-5.\\
        \hline
        11& \textbf{An eigener Waffe verletzt: INI -3}. Waffenschaden (ohne TP/KK oder Ansage) inkl. Wunden falls über KO/2 SP.\\
        \hline
        12& \textbf{Schwerer Eigentreffer: INI -4}. Doppelter Waffenschaden (ohne TP/KK oder Ansage) inkl. Wunden.\\
        \hline
    \end{tabularx}\\
\end{center}
\subsection{Patzer}
\begin{center}
    \begin{tabularx}{\linewidth}{cX}
        \hline
        \textbf{2W6} & \textbf{Auswirkungen}\\
        \hline
        ans <= BF & Waffe/Schild des Verteidigers zerbricht\\
        \hline
        12 > ans > BF & BF von Waffe/Schild des Verteidigers steigt um 1.\\
        \hline
        ans == 12 & BF von Waffe/Schild des Verteidigers unverändert; Angreifer muss Bruchtest ablegen.\\
        \hline
    \end{tabularx}
\end{center}
}{\section{Fernkampf}

\begin{center}
    \begin{tabularx}{\linewidth}{cX}
        \hline
        \textbf{2W6}& \textbf{Auswirkungen}\\
        \hline
        2& \textbf{Waffe zerstört: INI -4}. Hässliches Knacken, Waffe so schwer
        beschädigt, dass Reparatur nicht lohnt. \textbf{Schütze verliert alle
        verbleibenden Aktionen dieser Runde.}\\
        \hline
        3& \textbf{Waffe beschädigt: INI -3}. Projektil landet vor Füßen des
        Schützen. Sehne ist gerissen, Armbrust ernsthaft verklemmt.
        \textbf{Schütze verliert alle verbleibenden Aktionen dieser Runde.}\\
        \hline
        4-10& \textbf{Fehlschuss: INI -2}. Schütze benötigt \textbf{2 Aktionen} um Waffe wieder schussfähig zu machen.\\
        \hline
        11-12& \textbf{Kamerade getroffen: INI -3}. Der Schuss löst sich
        unbeabwsichtigt. Trifft den nächsten Helden in der Schussbahn.
        \textbf{Ansagen} kommen nicht zum tragen beim Schaden. \textbf{Ist
        kein Held in der nähe, hat Schütze sich selbst verletzt.} (TP gemäß
        geringster Entfernung.)\\
        \hline
    \end{tabularx}
\end{center}
}
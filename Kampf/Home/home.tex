\chapter{Kampfablauf}
\spp{
\section{Kampfbeginn}
\begin{center}
    \begin{tikzpicture}[node distance = 3cm, auto]
            % Nodes
        \node [cloud] (init) {Start} ;
        \node [block, below of=init] (initiative) {\hypertarget{initiativestart}{\hylin{initiative}{Initiative}}};
        \node [block, below of=initiative] (ansage) {Ansage von niedrigster Initiative an};
        \node [block, right of=ansage] (hinweis) {Manöver mit 2 Aktionen nicht umwandelbar};
        \node [block, below of=ansage] (ausf) {Ausführung von höchster Initiative an};
        \node [block, below of=ausf] (angr) {\hypertarget{angriffstart}{\hylin{angriff}{Angriff}}};
        \node [block, below of=angr] (par) {\hylt{paradestart}{parade}{Parade}};
            % Edges
        \path [line] (init) -> (initiative);
        \path [line] (initiative) -> (ansage);
        \path [line] (ansage) -> (ausf);
        \path [line] (ausf) -> (angr);
        \path [line] (angr) -> (par);
    \end{tikzpicture}
\end{center}
}{
    \section{Ablauf eines Initiativeschritts}
    \begin{center}
        \begin{tikzpicture}[node distance = 3cm, auto]
                % Nodes
            \node [cloud] (init) {Du bist dran};
            \node [decision, below of=init] (aufmerk) {Aufm.?};
            \node [decision, below left of=aufmerk] (ans) {Abwar.?};
            \node [block, below left of=ans] (abw) {\hylin{abwarten}{Abwarten}};
            \node [block, below right of=ans] (aend) {Ansage ändern};
            \node [block, below of=aend] (ausf) {Ausführen};
            \node [below right = 0.5cm of aufmerk] (no) {Nein};
                % Edges
            \path [line] (init) -- (aufmerk);
            \path [line] (aufmerk) --node[above left] {Ja} (ans);
            \path [line] (ans) --node[above left] {Ja} (abw);
            \draw [line] (aufmerk) to [bend left=40] (ausf);
            \path [line] (ans) --node[above right] {Nein} (aend);
            \path [line] (aend) -- (ausf);
        \end{tikzpicture}
    \end{center}
}
\newpage
\section{Initiative / Hinterhalt}
\begin{minipage}[t]{\linewidth}
\begin{center}
    \begin{tikzpicture}[node distance = 3cm, auto]
            % Nodes
        \node [cloud] (init) {\hypertarget{initiative}{\hylin{initiativestart}{Initiative}}} ;
        \node [decision, below of=init] (surpr) {Überrascht};
        \node [decision, below left of=surpr] (surprja) {Hinterhalt};
        \node [block, below right of=surpr] (surprno) {Initiative Wurf};
        \node [block, below right of=surprja] (hintno) {\textbf{IN}/KR für Ini Wurf*};
        \node [block, below left of=surprja] (hintja) {Freie Handlung f. Angreifenden};
        \node [block, below of=hintja] (hintausw) {\textbf{IN}(+3 bei Fernkampf) zum Parieren/Ausw.};
            % Edges
        \path [line] (init) -- (surpr);
        \path [line] (surpr) -- node[above left] {Ja} (surprja);
        \path [line] (surpr) -- node {Nein} (surprno);
        \path [line] (surprja) -- node {Nein} (hintno);
        \path [line] (surprja) -- node[above left] {Ja} (hintja);
        \path [line] (hintja) -- (hintausw);
        \draw [line]    (hintausw) to[out=0,in=-90] (hintno);
    \end{tikzpicture}
\end{center}
\end{minipage}
\spp{
    \begin{center}
        \begin{tabularx}{\linewidth}{|c|X|}
            \multicolumn{2}{l}{\textbf{*)} Erleichterungen der Probe zu KR Beginn}\\
            \hline
           -1 / 2 TaP &  \textit{Kriegskunst}, \textit{Gefahreninstinkt}\\
           \hline
           -4 & \textit{Aufmerksamkeit}, \textit{Kampfgespür}\\
           \hline
        \end{tabularx}
    \end{center}
}{
    Auf Aktion \textit{Waffe ziehen} achten!
}
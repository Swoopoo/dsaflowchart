\chapter{Aktionen}
\spp{
\section{Abwarten}
\hypertarget{abwarten}{}
Man muss unterscheiden zwischen \textit{verzögern} und \textit{abwarten}.

\textbf{Abwarten} kann \textit{nur} derjenige mit der höheren INI. Sollten beide Parteien eines Zweikampfes verzögern:

In dem Fall nehmen beide ihre aufgesparte Aktion mit in die nächste Runde, zu
Beginn der Runde (also noch bevor einer der beiden mit der INI dran ist)
attackiert derjenige mit der höheren INI (muss er natürlich nicht, aber
jedenfalls ist dies die kluge Vorgehensweise). Der mit der niedrigeren INI
kann sich dann entscheiden seine aufgesparte Aktion zu verlieren und seine
Parade dieser Runde zu verwenden oder er kriegt einen unparierbaren Angriff,
wenn der trifft muss er eine \textit{Selbstbeherrschungs}-Probe + SP hinlegen oder er
\textbf{verliert} seine aufgesparte Aktion.

\mbox{}
\vfill
\hline
\mbox{}
\vfill
\textbf{*)} Ausführender selbst zählt \textbf{nicht} als Beteiligter!

\textbf{**)} Freiwillig um doppelte Überzahl der Gegner erschwert, dann fällt Überzahlbonus weg.

\textbf{***)} Max. 6 auf INI Werte der Beteiligten (\textbf{nicht} er selbst), wenn \textbf{positiver} Kriegskunstwert.
}{
    \section{Taktik}
    \textbf{Einmal} pro KR und einmalig pro Beteiligtem!
\begin{center}
    \begin{tikzpicture}[node distance = 2cm, auto]
            % Nodes
        \node [block] (init) {Anführer (höchste Kriegskunst)};
        \node [decision, below of=init] (ueberr) {Überra.?};
        \node [right of=ueberr] {Nein};
        \node [block, below = 1cm of ueberr] (in) {Erst nach \textbf{IN} Probe};
        \node [block, below of=in] (dec) {Festlegen der Beteiligten *};
        \node [decision, below of=dec] (kk) {\textit{Kriegskunst} Probe**};
        \node [block, below left = 1cm of kk] (jo) {\textbf{TaP*} gleichmäßig ***};
        \node [block, below right = 1cm of kk] (no) {-1 INI für jeden \textbf{Beteiligten}};
            % Edges
        \path [line] (init) -- (ueberr);
        \path [line] (ueberr) --node[left] {Ja} (in);
        \draw [line] (ueberr) to [bend left=60] (dec);
        \path [line] (in) -- (dec);
        \path [line] (dec) -- (kk);
        \path [line] (kk) --node [above left] {Ja} (jo);
        \path [line] (kk) --node {Nein} (no);
    \end{tikzpicture}
\end{center}
}
\newpage
\section{Meucheln / Betäuben}
Vorraussetzungen: 
\begin{enumerate}
    \item Kommt an Opfer unbemerkt heran (\textit{Schleichen} / \textit{Sich Verstecken})
    \item passende Waffe in der Hand, die \textbf{Opfer nicht bemerkt hat}
    \item Opfer muss \textbf{völlig} ahnungslos sein
\end{enumerate}
\begin{center}
    \begin{tikzpicture}[node distance = 2cm, auto]
            % Nodes
        \node [block] (init) {Opfer ahnungslos};
        \node [block, below of=init] (hit) {Attacke*};
        \node [decision, below of=hit] (vorn) {Von vorne?};
        \node [decision, left of=vorn] (vornjo) {\textbf{IN} Probe Opfer};
        \node [decision, below = 1cm of vornjo] (inpro) {\textit{Raufen}-PA +8};
        \node [block, below = 1cm of vorn] (treffer) {Schaden};
        \node [decision, right = 2cm of treffer] (toeten) {Töten?};
        \node [block, above = 1cm of toeten] (tp) {TP = SP; SP x3**};
        \node [block, right = 1cm of toeten] (bet) {TP = SP(A)};
        \node [decision, above of=bet] (wund) {AU-Verl. > KO/2};
        \node [decision, above of=wund] (wund2) {AU-Verl. > KO};
        \node [block, right = 1cm of wund2] (end) {1W6 SR bewusstlos};
        \node [decision, below = 1cm of end] (prob) {\textbf{KO} + SP(A)};
            % Edges
        \path [line] (init) -- (hit);
        \path [line] (hit) -- (vorn);
        \path [line] (vorn) --node[above] {Ja} (vornjo);
        \path [line] (vorn) --node[right] {Nein} (treffer);
        \path [line] (vornjo) --node[above right] {Nein} (treffer);
        \path [line] (vornjo) --node[left] {Ja} (inpro);
        \path [line] (inpro) --node[above] {Nein} (treffer);
        \path [line] (treffer) -- (toeten);
        \path [line] (toeten) --node[above] {Nein} (bet);
        \path [line] (toeten) --node[left] {Ja} (tp);
        \path [line] (bet) -- (wund);
        \path [line] (wund) --node[left] {Ja} (wund2);
        \path [line] (wund) --node[below] {Nein} (end);
        \path [line] (wund2) --node[above] {Ja} (prob);
        \path [line] (prob) --node[left] {Nein} (end);


    \end{tikzpicture}
\end{center}
\textbf{*)} Je nach Situation um 5 (ahnungslos) bzw. 8 (schläft, bewusstlos,
gefesselt). Mit \textit{Gezielter Stich} um bis zu 8 erschwerbar.
\textit{Anatomie}-Probe: Voller Schaden (wie 6 gewürfelt). Bei
\textbf{Trefferzone} mit \textit{Gezielter Stich} nur halber Aufschlag.

\textbf{**)} Nur mit Dolch, Fechtwaffen und Schwertern möglich.